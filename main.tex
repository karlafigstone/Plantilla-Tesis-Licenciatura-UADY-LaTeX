%----------------------------------------------------------
% Autor: Karla A. Figueroa
% email: karlafigueroa186@gmail.com

% Comparto la plantilla del documento que hice para escribir
% mi tesis de licenciatura.
% Cumple con los requisitos de la Facultad de Matemáticas
% de la UADY: páginas a una sola cara, numeración romana
% hasta antes de la introduccion, datos de portada, etc.
% Agregué paquetes y algunos comentarios de su función,
% asi como ejemplos de su uso.
%---------------------------------------------------------
\documentclass[letterpaper,12pt]{report}
\usepackage[utf8]{inputenc}
\usepackage[top=2.5cm, left=3cm, right=2.5cm, bottom=2.5cm]{geometry}
\renewcommand{\baselinestretch}{1.5} % espaciado 1.5
\parindent = 0mm % Sin sangría
\usepackage[T1]{fontenc}
% --- Paquetes de mates ---
\usepackage{mathptmx}
\usepackage{mathrsfs}
\usepackage{mathtools}
\usepackage{latexsym,amssymb,amsfonts,amsthm}
\usepackage{amsmath}
% --- Imágenes ---
\usepackage{graphicx} % insertar imégenes
\usepackage{subfig} % insertar varias imagenes
% --- Formato ---
\usepackage{color}
\usepackage{array}
\usepackage{multirow} % combina celdas
\usepackage{xcolor, colortbl}
\usepackage[spanish, es-lcroman, es-tabla]{babel}\decimalpoint % Idioma español
\usepackage{caption} % Paquete para personalizar captions
\captionsetup{font=footnotesize} % Configura el tamaño de la fuente para captions
\usepackage{tabularx}
\usepackage{xfrac}
% --- Estilo ---
\usepackage{fancyhdr}
\fancypagestyle{plain}{
\fancyhf{} 
\fancyfoot[R]{\thepage}} %enumera las páginas a la derecha
\pagestyle{fancy}
\fancyhf{}
\fancyfoot[R]{\thepage}
\renewcommand{\headrulewidth}{0pt} %quita la linea del encabezado de pagina
% --- Estilo para resultados ---
\theoremstyle{plain}
\newtheorem{teorema}{Teorema}[section]
\newtheorem{lema}[teorema]{Lema}
\newtheorem{proposicion}[teorema]{Proposición}
\newtheorem{corolario}[teorema]{Corolario}
% --- Estilo para definiciones, etc ---
\theoremstyle{definicion}
\newtheorem{definicion}[teorema]{Definición}
\newtheorem{ejemplo}[teorema]{Ejemplo}
\newtheorem{observacion}[teorema]{Observación}
\renewcommand{\proofname}{Demostración} %cambia el nombre a demostracion

\usepackage{natbib}
\usepackage{hyperref} %permite poner hiperreferencias o links
\usepackage[nottoc]{tocbibind} %agrega la bibliografia al indice


% --- DOCUMENTO ---
% se modifica el contenido en cada archivo .tex que se
% encuentra dentro de la carpeta 'capitulos'
% si no necesitas aguna sección solo debes comentarla
\begin{document}

% --- Portada ---
% debes agregar tus datos en el archivo portada.tex
\begin{titlepage}
\thispagestyle{empty}
\begin{center}

% --- ESCUDO UADY ---
%\includegraphics[scale=0.4]{escudo-uady.PNG}\\[0.3cm]
\begin{center}
    \fbox{
        \parbox[c][6cm][c]{4cm}{
            \centering
            LOGO \\
            AQUÍ
        }}\\[0.5cm]
\end{center}
% por derechos de autor no incluyo la imagen del escudo pero
% puedes incluirla como 'escudo-uady' y quitar el recuadro
% --- UNIVERSIDAD Y FACULTAD ---
{\Large \MakeUppercase{Universidad Aut\'onoma de Yucatán}}\\
{\large \MakeUppercase{Facultad de Matem\'aticas}}\\
{\large \MakeUppercase{Licenciatura en Matem\'aticas}}\\[1cm]
% --- TÍTULO DE LA TESIS ---
\MakeUppercase{\large {Commodo ipsum ad nisi eiusmod exercitation eiusmod exercitation sit consectetur}}\\[1.2cm]
% --- T E S I S ---
{\huge T \hspace{1cm} E \hspace{1cm} S \hspace{1cm} I \hspace{1cm} S}\\[1.2cm]
% --- GRADO Y AUTOR ---
\MakeUppercase{Que para obtener el t\'itulo de:}\\
\MakeUppercase{\large{Licenciado(a) en Matem\'aticas}}\\[0.5cm]
\MakeUppercase{Presenta:}\\
\MakeUppercase{\large{Alumno de la universidad}}\\[0.5cm]
% --- ASESOR ---
\MakeUppercase{Director de tesis:}\\
\MakeUppercase{\large{Dr. Asesor de tesis}}\\[1.2cm]
% --- LUGAR Y AÑO ---
\MakeUppercase{M\'erida, Yucat\'an, 2025}\\

\end{center}
\end{titlepage}

% --- Resumen, dedicatoria, agradecimientos ---
\pagenumbering{roman}
\chapter*{\color{white}Dedicatoria}
%\addcontentsline{toc}{chapter}{Dedicatoria} %esto es para que aparezca en el índice
\phantomsection
\begin{flushright}
    \textit{Dedicatoria o ep\'igrafe}
\end{flushright}

\newpage
\chapter*{Agradecimientos}
\addcontentsline{toc}{chapter}{Agradecimientos}
Lorem ipsum dolor sit amet, consectetur adipiscing elit, sed do eiusmod tempor incididunt ut labore et dolore magna aliqua. Ut enim ad minim veniam, quis nostrud exercitation ullamco laboris nisi ut dixun ex ea commodo consecuencia. Duis aute irure dolor in reprehenderit in voluptate velit esse cillum dolore eu fugiat nulla pariatur. Excepteur sint occaecat cupidatat non proident, sunt in culpa qui officia deserunt mollit anim id est laborum.\\


\newpage
\chapter*{Resumen}
\addcontentsline{toc}{chapter}{Resumen}
Lorem ipsum dolor sit amet, consectetur adipiscing elit, sed do eiusmod tempor incididunt ut labore et dolore magna aliqua. Ut enim ad minim veniam, quis nostrud exercitation ullamco laboris nisi ut dixun ex ea commodo consecuencia. Duis aute irure dolor in reprehenderit in voluptate velit esse cillum dolore eu fugiat nulla pariatur. Excepteur sint occaecat cupidatat non proident, sunt in culpa qui officia deserunt mollit anim id est laborum.\\

Lorem ipsum dolor sit amet, consectetur adipiscing elit, sed do eiusmod tempor incididunt ut labore et dolore magna aliqua. Ut enim ad minim veniam, quis nostrud exercitation ullamco laboris nisi ut dixun ex ea commodo consecuencia. Duis aute irure dolor in reprehenderit in voluptate velit esse cillum dolore eu fugiat nulla pariatur. Excepteur sint occaecat cupidatat non proident, sunt in culpa qui officia deserunt mollit anim id est laborum.\\

% --- Tabla de contenidos ---
\hypersetup{pdfborder=0 0 0} %quita los recuadros del vinculo
\renewcommand*\contentsname{\'Indice general} %renombra el índice
\tableofcontents %crea el índice
\listoffigures
\listoftables

% --- Cuerpo de la tesis ---
\include{capitulos/introduccion}
\include{capitulos/objetivo}
\chapter{Marco te\'orico} %\label{mteorico}
Puedes citar de la siguiente manera: \cite{Boyce1965,Currie2012, Groetsch1993,Stewart2008}. Lo que cites aparecerá en la bibliografía.

\section{Definiciones, teoremas y demostraciones}
\begin{definicion}
Una funci\'on $f$ es lipschitziana si...
\end{definicion}

\begin{teorema}
Sea $f$ continua en $[a,b]$. Entonces...
\end{teorema}

\begin{proof}
Aqu\'i va la demostraci\'on.\\
Como $f$ es continua...
\end{proof}

\begin{ejemplo}
La funci\'on $f(x)=|x|$ es lipschitziana.
\end{ejemplo}

\section{Agregar ecuaciones}

Para agregar ecuaciones
\begin{equation}\label{cons_moment}
    \frac{\partial u}{\partial t}+u\cdot\nabla u=\frac{\mu}{\rho}\nabla^{2} u-\frac{1}{\rho}\nabla p+f,
\end{equation}
Para hacer referencia a la Ecuaci\'on \eqref{cons_moment} poniendole una etiqueta (label). Si no quieres enumerarla usa un asterisco en el entorno
\begin{equation*}
    \frac{\partial u}{\partial t}+u\cdot\nabla u=\frac{\mu}{\rho}\nabla^{2} u-\frac{1}{\rho}\nabla p+f,
\end{equation*}

Si la ecuaci\'on es muy larga la puedes dividir
\begin{multline}\label{osc_comp2}
    \sin^{2}{\theta}+ 2\sin{\theta}\cos{\theta} +\frac{\sin{\theta}\sin{\phi}} +\left(\frac{2\pi}{2}\right)^{2}(\sin{\phi}\cos{\phi}\sin^{2}{\theta})\\ +\frac{2}{\pi}(\sin{\phi}+\cos{\phi}) + 2\sin{\theta}\cos{\theta} =0.
\end{multline}

\section{Agregar tablas y combinar celdas}

\begin{table}[h!] % h indica 'here' es la ubicacion de la tabla
    \centering
    \caption{Titulo de la tabla}
    \begin{tabularx}{0.4\textwidth}{cccc} % c es la columna
        \hline
         & Ramas & $a_4$ & $a_3$\\ \hline
        \multirow{2}{*}{$x$} & A & $0$ & $0$ \\
        &B & $1\times10^{-4}$ & $-1\times10^{-4}$ \\ \hline
    \end{tabularx}
    \label{tab:tabla1}
\end{table}
Para hacer referencia a la Tabla \ref{tab:tabla1}

\section{Agregar im\'agenes y grupos de imágenes}
\begin{figure}[ht]
   \centering
   \includegraphics[scale=0.2]{figuras/ejemplo1.PNG}
   \caption{Figura1}
   \label{fig:figura1}
\end{figure}

\begin{figure}[h!]
    \centering
    \subfloat[Subfigura1]{
    \includegraphics[scale=0.2]{figuras/identidad.PNG}
    \label{fig:ineal}}
    \subfloat[Subfigura2]{
    \includegraphics[scale=0.2]{figuras/parabola.PNG}
    \label{fig:parab}} 
    \caption{Figura2}
   \label{fig:figura2}
\end{figure}
Puedes hacer referencia a la Figura \ref{fig:figura1} o a la Figura \ref{fig:ineal}.
\include{capitulos/metodologia}
\include{capitulos/resultados}
\include{capitulos/concluisones}
\include{capitulos/apendice}

% --- Bibliografía ---
\bibliographystyle{plainnat} %las referencias con nombre
%\bibliographystyle{unsrt} %las referencias con número
\bibliography{referencias} %aqui pones tu archivo .bib

\end{document}