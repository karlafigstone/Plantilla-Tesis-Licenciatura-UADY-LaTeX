\chapter{Marco te\'orico} %\label{mteorico}
Puedes citar de la siguiente manera: \cite{Boyce1965,Currie2012, Groetsch1993,Stewart2008}. Lo que cites aparecerá en la bibliografía.

\section{Definiciones, teoremas y demostraciones}
\begin{definicion}
Una funci\'on $f$ es lipschitziana si...
\end{definicion}

\begin{teorema}
Sea $f$ continua en $[a,b]$. Entonces...
\end{teorema}

\begin{proof}
Aqu\'i va la demostraci\'on.\\
Como $f$ es continua...
\end{proof}

\begin{ejemplo}
La funci\'on $f(x)=|x|$ es lipschitziana.
\end{ejemplo}

\section{Agregar ecuaciones}

Para agregar ecuaciones
\begin{equation}\label{cons_moment}
    \frac{\partial u}{\partial t}+u\cdot\nabla u=\frac{\mu}{\rho}\nabla^{2} u-\frac{1}{\rho}\nabla p+f,
\end{equation}
Para hacer referencia a la Ecuaci\'on \eqref{cons_moment} poniendole una etiqueta (label). Si no quieres enumerarla usa un asterisco en el entorno
\begin{equation*}
    \frac{\partial u}{\partial t}+u\cdot\nabla u=\frac{\mu}{\rho}\nabla^{2} u-\frac{1}{\rho}\nabla p+f,
\end{equation*}

Si la ecuaci\'on es muy larga la puedes dividir
\begin{multline}\label{osc_comp2}
    \sin^{2}{\theta}+ 2\sin{\theta}\cos{\theta} +\frac{\sin{\theta}\sin{\phi}} +\left(\frac{2\pi}{2}\right)^{2}(\sin{\phi}\cos{\phi}\sin^{2}{\theta})\\ +\frac{2}{\pi}(\sin{\phi}+\cos{\phi}) + 2\sin{\theta}\cos{\theta} =0.
\end{multline}

\section{Agregar tablas y combinar celdas}

\begin{table}[h!] % h indica 'here' es la ubicacion de la tabla
    \centering
    \caption{Titulo de la tabla}
    \begin{tabularx}{0.4\textwidth}{cccc} % c es la columna
        \hline
         & Ramas & $a_4$ & $a_3$\\ \hline
        \multirow{2}{*}{$x$} & A & $0$ & $0$ \\
        &B & $1\times10^{-4}$ & $-1\times10^{-4}$ \\ \hline
    \end{tabularx}
    \label{tab:tabla1}
\end{table}
Para hacer referencia a la Tabla \ref{tab:tabla1}

\section{Agregar im\'agenes y grupos de imágenes}
\begin{figure}[ht]
   \centering
   \includegraphics[scale=0.2]{figuras/ejemplo1.PNG}
   \caption{Figura1}
   \label{fig:figura1}
\end{figure}

\begin{figure}[h!]
    \centering
    \subfloat[Subfigura1]{
    \includegraphics[scale=0.2]{figuras/identidad.PNG}
    \label{fig:ineal}}
    \subfloat[Subfigura2]{
    \includegraphics[scale=0.2]{figuras/parabola.PNG}
    \label{fig:parab}} 
    \caption{Figura2}
   \label{fig:figura2}
\end{figure}
Puedes hacer referencia a la Figura \ref{fig:figura1} o a la Figura \ref{fig:ineal}.